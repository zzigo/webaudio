1
Five Things We Need to Know About Technological Change
by
Neil Postman
Talk delivered in Denver Colorado
March 28, 1998

… I doubt that the 21st century will pose for us problems that are more stunning, disorienting or complex
than those we faced in this century, or the 19th, 18th, 17th, or for that matter, many of the centuries before
that. But for those who are excessively nervous about the new millennium, I can provide, right at the start,
some good advice about how to confront it. …. Here is what Henry David Thoreau told us: “All our
inventions are but improved means to an unimproved end.” Here is what Goethe told us: “One should, each
day, try to hear a little song, read a good poem, see a fine picture, and, if possible, speak a few reasonable
words.” Socrates told us: “The unexamined life is not worth living.” Rabbi Hillel told us: “What is hateful
to thee, do not do to another.” And here is the prophet Micah: “What does the Lord require of thee but to do
justly, to love mercy and to walk humbly with thy God.” And I could say, if we had the time, (although you
know it well enough) what Jesus, Isaiah, Mohammad, Spinoza, and Shakespeare told us. It is all the same:
There is no escaping from ourselves. The human dilemma is as it has always been, and it is a delusion to
believe that the technological changes of our era have rendered irrelevant the wisdom of the ages and the
sages.

Nonetheless, having said this, I know perfectly well that because we do live in a technological age, we have
some special problems that Jesus, Hillel, Socrates, and Micah did not and could not speak of. I do not have
the wisdom to say what we ought to do about such problems, and so my contribution must confine itself to
some things we need to know in order to address the problems. I call my talk Five Things We Need to
Know About Technological Change. I base these ideas on my thirty years of studying the history of
technological change but I do not think these are academic or esoteric ideas. They are to the sort of things
everyone who is concerned with cultural stability and balance should know and I offer them to you in the
hope that you will find them useful in thinking about the effects of technology on religious faith.

First Idea
The first idea is that all technological change is a trade-off. I like to call it a Faustian bargain. Technology
giveth and technology taketh away. This means that for every advantage a new technology offers, there is
always a corresponding disadvantage. The disadvantage may exceed in importance the advantage, or the
advantage may well be worth the cost. Now, this may seem to be a rather obvious idea, but you would be
surprised at how many people believe that new technologies are unmixed blessings. You need only think of
the enthusiasms with which most people approach their understanding of computers. Ask anyone who
knows something about computers to talk about them, and you will find that they will, unabashedly and
relentlessly, extol the wonders of computers. You will also find that in most cases they will completely
neglect to mention any of the liabilities of computers. This is a dangerous imbalance, since the greater the
wonders of a technology, the greater will be its negative consequences.
Think of the automobile, which for all of its obvious advantages, has poisoned our air, choked our cities,
and degraded the beauty of our natural landscape. Or you might reflect on the paradox of medical
technology which brings wondrous cures but is, at the same time, a demonstrable cause of certain diseases
and disabilities, and has played a significant role in reducing the diagnostic skills of physicians. It is also
well to recall that for all of the intellectual and social benefits provided by the printing press, its costs were
equally monumental. The printing press gave the Western world prose, but it made poetry into an exotic
and elitist form of communication. It gave us inductive science, but it reduced religious sensibility to a
form of fanciful superstition. Printing gave us the modern conception of nationhood, but in so doing turned
patriotism into a sordid if not lethal emotion. We might even say that the printing of the Bible in vernacular 

2
languages introduced the impression that God was an Englishman or a German or a Frenchman—that is to
say, printing reduced God to the dimensions of a local potentate.
Perhaps the best way I can express this idea is to say that the question, “What will a new technology do?” is
no more important than the question, “What will a new technology undo?” Indeed, the latter question is
more important, precisely because it is asked so infrequently. One might say, then, that a sophisticated
perspective on technological change includes one’s being skeptical of Utopian and Messianic visions drawn
by those who have no sense of history or of the precarious balances on which culture depends. In fact, if it
were up to me, I would forbid anyone from talking about the new information technologies unless the
person can demonstrate that he or she knows something about the social and psychic effects of the alphabet,
the mechanical clock, the printing press, and telegraphy. In other words, knows something about the costs
of great technologies.
Idea Number One, then, is that culture always pays a price for technology.


Second Idea
This leads to the second idea, which is that the advantages and disadvantages of new technologies are never
distributed evenly among the population. This means that every new technology benefits some and harms
others. There are even some who are not affected at all. Consider again the case of the printing press in the
16th century, of which Martin Luther said it was “God’s highest and extremest act of grace, whereby the
business of the gospel is driven forward.” By placing the word of God on every Christian’s kitchen table,
the mass-produced book undermined the authority of the church hierarchy, and hastened the breakup of the
Holy Roman See. The Protestants of that time cheered this development. The Catholics were enraged and
distraught. Since I am a Jew, had I lived at that time, I probably wouldn’t have given a damn one way or
another, since it would make no difference whether a pogrom was inspired by Martin Luther or Pope Leo X.
Some gain, some lose, a few remain as they were.
Let us take as another example, television, although here I should add at once that in the case of television
there are very few indeed who are not affected in one way or another. In America, where television has
taken hold more deeply than anywhere else, there are many people who find it a blessing, not least those
who have achieved high-paying, gratifying careers in television as executives, technicians, directors,
newscasters and entertainers. On the other hand, and in the long run, television may bring an end to the
careers of school teachers since school was an invention of the printing press and must stand or fall on the
issue of how much importance the printed word will have in the future. There is no chance, of course, that
television will go away but school teachers who are enthusiastic about its presence always call to my mind
an image of some turn-of-the-century blacksmith who not only is singing the praises of the automobile but
who also believes that his business will be enhanced by it. We know now that his business was not
enhanced by it; it was rendered obsolete by it, as perhaps an intelligent blacksmith would have known.
The questions, then, that are never far from the mind of a person who is knowledgeable about technological
change are these: Who specifically benefits from the development of a new technology? Which groups,
what type of person, what kind of industry will be favored? And, of course, which groups of people will
thereby be harmed?


These questions should certainly be on our minds when we think about computer technology. There is no
doubt that the computer has been and will continue to be advantageous to large-scale organizations like the
military or airline companies or banks or tax collecting institutions. And it is equally clear that the
computer is now indispensable to high-level researchers in physics and other natural sciences. But to what
extent has computer technology been an advantage to the masses of people? To steel workers, vegetable
store owners, automobile mechanics, musicians, bakers, bricklayers, dentists, yes, theologians, and most of
the rest into whose lives the computer now intrudes? These people have had their private matters made
more accessible to powerful institutions. They are more easily tracked and controlled; they are subjected to
more examinations, and are increasingly mystified by the decisions made about them. They are more than 


3
ever reduced to mere numerical objects. They are being buried by junk mail. They are easy targets for
advertising agencies and political institutions.
In a word, these people are losers in the great computer revolution. The winners, which include among
others computer companies, multi-national corporations and the nation state, will, of course, encourage the
losers to be enthusiastic about computer technology. That is the way of winners, and so in the beginning
they told the losers that with personal computers the average person can balance a checkbook more neatly,
keep better track of recipes, and make more logical shopping lists. Then they told them that computers will
make it possible to vote at home, shop at home, get all the entertainment they wish at home, and thus make
community life unnecessary. And now, of course, the winners speak constantly of the Age of Information,
always implying that the more information we have, the better we will be in solving significant problems—
not only personal ones but large-scale social problems, as well. But how true is this? If there are children
starving in the world—and there are—it is not because of insufficient information. We have known for a
long time how to produce enough food to feed every child on the planet. How is it that we let so many of
them starve? If there is violence on our streets, it is not because we have insufficient information. If women
are abused, if divorce and pornography and mental illness are increasing, none of it has anything to do with
insufficient information. I dare say it is because something else is missing, and I don’t think I have to tell
this audience what it is. Who knows? This age of information may turn out to be a curse if we are blinded
by it so that we cannot see truly where our problems lie. That is why it is always necessary for us to ask of
those who speak enthusiastically of computer technology, why do you do this? What interests do you
represent? To whom are you hoping to give power? From whom will you be withholding power?
I do not mean to attribute unsavory, let alone sinister motives to anyone. I say only that since technology
favors some people and harms others, these are questions that must always be asked. And so, that there are
always winners and losers in technological change is the second idea.


Third Idea
Here is the third. Embedded in every technology there is a powerful idea, sometimes two or three powerful
ideas. These ideas are often hidden from our view because they are of a somewhat abstract nature. But this
should not be taken to mean that they do not have practical consequences.
Perhaps you are familiar with the old adage that says: To a man with a hammer, everything looks like a nail.
We may extend that truism: To a person with a pencil, everything looks like a sentence. To a person with a
TV camera, everything looks like an image. To a person with a computer, everything looks like data. I do
not think we need to take these aphorisms literally. But what they call to our attention is that every
technology has a prejudice. Like language itself, it predisposes us to favor and value certain perspectives
and accomplishments. In a culture without writing, human memory is of the greatest importance, as are the
proverbs, sayings and songs which contain the accumulated oral wisdom of centuries. That is why Solomon
was thought to be the wisest of men. In Kings I we are told he knew 3,000 proverbs. But in a culture with
writing, such feats of memory are considered a waste of time, and proverbs are merely irrelevant fancies.
The writing person favors logical organization and systematic analysis, not proverbs. The telegraphic
person values speed, not introspection. The television person values immediacy, not history. And computer
people, what shall we say of them? Perhaps we can say that the computer person values information, not
knowledge, certainly not wisdom. Indeed, in the computer age, the concept of wisdom may vanish
altogether.
The third idea, then, is that every technology has a philosophy which is given expression in how the
technology makes people use their minds, in what it makes us do with our bodies, in how it codifies the
world, in which of our senses it amplifies, in which of our emotional and intellectual tendencies it
disregards. This idea is the sum and substance of what the great Catholic prophet, Marshall McLuhan
meant when he coined the famous sentence, “The medium is the message.”


Fourth Idea
4
Here is the fourth idea: Technological change is not additive; it is ecological. I can explain this best by an
analogy. What happens if we place a drop of red dye into a beaker of clear water? Do we have clear water
plus a spot of red dye? Obviously not. We have a new coloration to every molecule of water. That is what I
mean by ecological change. A new medium does not add something; it changes everything. In the year
1500, after the printing press was invented, you did not have old Europe plus the printing press. You had a
different Europe. After television, America was not America plus television. Television gave a new
coloration to every political campaign, to every home, to every school, to every church, to every industry,
and so on.

That is why we must be cautious about technological innovation. The consequences of technological
change are always vast, often unpredictable and largely irreversible. That is also why we must be
suspicious of capitalists. Capitalists are by definition not only personal risk takers but, more to the point,
cultural risk takers. The most creative and daring of them hope to exploit new technologies to the fullest,
and do not much care what traditions are overthrown in the process or whether or not a culture is prepared
to function without such traditions. Capitalists are, in a word, radicals. In America, our most significant
radicals have always been capitalists—men like Bell, Edison, Ford, Carnegie, Sarnoff, Goldwyn. These
men obliterated the 19th century, and created the 20th, which is why it is a mystery to me that capitalists
are thought to be conservative. Perhaps it is because they are inclined to wear dark suits and grey ties.
I trust you understand that in saying all this, I am making no argument for socialism. I say only that
capitalists need to be carefully watched and disciplined. To be sure, they talk of family, marriage, piety,
and honor but if allowed to exploit new technology to its fullest economic potential, they may undo the
institutions that make such ideas possible. And here I might just give two examples of this point, taken
from the American encounter with technology. The first concerns education. Who, we may ask, has had the
greatest impact on American education in this century? If you are thinking of John Dewey or any other
education philosopher, I must say you are quite wrong. The greatest impact has been made by quiet men in
grey suits in a suburb of New York City called Princeton, New Jersey. There, they developed and promoted
the technology known as the standardized test, such as IQ tests, the SATs and the GREs. Their tests
redefined what we mean by learning, and have resulted in our reorganizing the curriculum to accommodate
the tests.
A second example concerns our politics. It is clear by now that the people who have had the most radical
effect on American politics in our time are not political ideologues or student protesters with long hair and
copies of Karl Marx under their arms. The radicals who have changed the nature of politics in America are
entrepreneurs in dark suits and grey ties who manage the large television industry in America. They did not
mean to turn political discourse into a form of entertainment. They did not mean to make it impossible for
an overweight person to run for high political office. They did not mean to reduce political campaigning to
a 30-second TV commercial. All they were trying to do is to make television into a vast and unsleeping
money machine. That they destroyed substantive political discourse in the process does not concern them.
Fifth Idea
I come now to the fifth and final idea, which is that media tend to become mythic. I use this word in the
sense in which it was used by the French literary critic, Roland Barthes. He used the word “myth” to refer
to a common tendency to think of our technological creations as if they were God-given, as if they were a
part of the natural order of things. I have on occasion asked my students if they know when the alphabet
was invented. The question astonishes them. It is as if I asked them when clouds and trees were invented.
The alphabet, they believe, was not something that was invented. It just is. It is this way with many
products of human culture but with none more consistently than technology. Cars, planes, TV, movies,
newspapers—they have achieved mythic status because they are perceived as gifts of nature, not as artifacts
produced in a specific political and historical context. 


5
When a technology become mythic, it is always dangerous because it is then accepted as it is, and is
therefore not easily susceptible to modification or control. If you should propose to the average American
that television broadcasting should not begin until 5 PM and should cease at 11 PM, or propose that there
should be no television commercials, he will think the idea ridiculous. But not because he disagrees with
your cultural agenda. He will think it ridiculous because he assumes you are proposing that something in
nature be changed; as if you are suggesting that the sun should rise at 10 AM instead of at 6.
Whenever I think about the capacity of technology to become mythic, I call to mind the remark made by
Pope John Paul II. He said, “Science can purify religion from error and superstition. Religion can purify
science from idolatry and false absolutes.”
What I am saying is that our enthusiasm for technology can turn into a form of idolatry and our belief in its
beneficence can be a false absolute. The best way to view technology is as a strange intruder, to remember
that technology is not part of God’s plan but a product of human creativity and hubris, and that its capacity
for good or evil rests entirely on human awareness of what it does for us and to us.


Conclusion
And so, these are my five ideas about technological change. First, that we always pay a price for
technology; the greater the technology, the greater the price. Second, that there are always winners and
losers, and that the winners always try to persuade the losers that they are really winners. Third, that there
is embedded in every great technology an epistemological, political or social prejudice. Sometimes that
bias is greatly to our advantage. Sometimes it is not. The printing press annihilated the oral tradition;
telegraphy annihilated space; television has humiliated the word; the computer, perhaps, will degrade
community life. And so on. Fourth, technological change is not additive; it is ecological, which means, it
changes everything and is, therefore, too important to be left entirely in the hands of Bill Gates. And fifth,
technology tends to become mythic; that is, perceived as part of the natural order of things, and therefore
tends to control more of our lives than is good for us.
If we had more time, I could supply some additional important things about technological change but I will
stand by these for the moment, and will close with this thought. In the past, we experienced technological
change in the manner of sleep-walkers. Our unspoken slogan has been “technology über alles,” and we
have been willing to shape our lives to fit the requirements of technology, not the requirements of culture.
This is a form of stupidity, especially in an age of vast technological change. We need to proceed with our
eyes wide open so that we many use technology rather than be used by it. 